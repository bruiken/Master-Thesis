\chapter{Evaluation}\label{sec:evaluation}
This chapter will cover the evaluation of the operators we designed and
formalised in Chapter~\ref{sec:vp:formalisation}. We will perform this
evaluation by looking at common editing operations in software systems and
seeing how well our system applies to these operations.

Note that we only referred to Chapter~\ref{sec:vp:formalisation} as to what this
chapter will cover. This is because we only cover the formalisation of the operators
here, not the implementation (from Chapter~\ref{ch:implementation}). The implementation
is not evaluated at large, given the scale of the thesis. The evaluation that \emph{is}
in place for the implementation are the unit tests created for it.

\section{Context and Methodology}
The editing operations we will look into are a collection of fourteen editing
operations, structured into three groups (code-adding, code-removing and other), 
that have been identified in~\cite{stuanciulescu2016}. These editing operations
have been extracted from commit data from the Marlin 3D printer firmware\footnote{\url{https://github.com/MarlinFirmware/}}.
This was a suitable repository as it is highly configurable and large (more than 140
features and 40,000 lines of code). The code-adding patterns are for adding variability (P1--P3),
partly wrapping existing code with variability (P4 and P5), adding code without variability (P6), and
repairing annotations (P7). Code-removing patterns exist for removing non annotated code (P8),
removing code with variability (P9), and for repairing annotations (P10). The other patterns 
include wrapping and unwrapping code with and from their variability (P11 and P12), changing
presence conditions of code (P13), and moving code between different variability blocks (P14).
Our evaluation is structured along the groups.

In this evaluation, we follow the same process as St{\u{a}}nciulescu et al.: we
take a look at every edit operation and see how our method supports them. Supporting in this
context means that we can fulfill the edit operation with our definitions of the \emph{get}
and \emph{put} operators. We will see that this comes down to picking suitable choices
and ambitions.
We keep the same naming scheme for the edit operations from the previous work, even
though they are centred around \texttt{\#ifdef} statements. This is acceptable, because
our presence conditions can be seen as \texttt{\#ifdef} statements and vice versa. This naming
scheme also eases comparability, which we will do after applying the edit operations.
For our system to be usable, we
want to be able to support all of the operations. We will also take a closer look 
at the restriction we placed on the \emph{put} operator, as we do not want it to
restrict us from using any common edit pattern. 

\section{Code-Adding Patterns}
\begin{figure}
  \centering
  \begin{tikzpicture}
    \treeone{OS}{0}{0}{$r_1$}{$a_1$}{}
    \treeone{OV}{0}{-2.5}{$r_2$}{$a_1$}{}
    \treetwo{EV}{3}{-2.5}{$r'_2$}{$a_1$}{$a_2$}{}{}
    \treetwo{ES}{3}{0}{$r_3$}{$a_1$}{$a_2$}{}{$A$}
    \path[] (OVAE) edge[double,-Stealth] node[midway, below] {Manual edit} (EVAW);
    \path[] (OSAS) edge[double,-Stealth] node[left] {$\mathit{get}\left(r_1, \mathit{true}\right)$} (OVAN);
    \path[] (EVAN) edge[double,-Stealth] node[right] {$\mathit{put}\left(r'_2, \left\{ a_2 \right\}, r_1, \mathit{true}, \mathit{A}\right)$} (ESAS);
  \end{tikzpicture}
  \caption{Workflow for adding a new asset, relevant for patterns \emph{P1} through \emph{P3}.}
  \label{fig:patterns:p1}
\end{figure}
\subsection*{P1 AddIfdef, P2 AddIfdef*, P3 AddIfdefElse}
The first three patterns are all grouped since they have a common goal:
adding new code with a presence condition. They were previously split up because
a distinction was made between adding one and multiple assets, and adding just an
\texttt{\#ifdef} and adding an \texttt{\#ifdef} combined with a \texttt{\#else} clause.
For us, adding new code always means adding a new asset. To do this, we can \emph{get}
the tree with a trivial choice (\(\truefeature\)) and \emph{put} the tree back with as
ambition the desired presence condition. We have to take care however, there is a limitation
on the \emph{put} operator. We cannot have any asset in the editing view that
has a presence condition which is not satisfiable with the ambition. In
particular, this means that we cannot have any asset $a$ in our view such that
\(\presenceconditionfunc\left(a\right) \land \mathit{ambition} \not\in \sat\).
This process can be seen in Figure~\ref{fig:patterns:p1}.

\subsection*{P4 AddIfdefWrapElse, P5 AddIfdefWrapThen}
\begin{figure}
  \centering
  \begin{tikzpicture}
    \treeone{OS}{0}{0}{$r_1$}{$a_1$}{}
    \treeone{OV}{0}{-2.5}{$r_2$}{$a_1$}{}
    \treeone{EV}{3}{-2.5}{$r'_2$}{$a_1$}{}
    \treetwo{ES}{3}{0}{$r_3$}{$a_1$}{$a'_1$}{$\neg A$}{$A$}
    \path[] (OVAE) edge[double,-Stealth] node[midway, below] {Manual edit} (EVAW);
    \path[] (OSAS) edge[double,-Stealth] node[left] {$\mathit{get}\left(r_1, \mathit{true}\right)$} (OVAN);
    \path[] (EVAN) edge[double,-Stealth] node[right] {$\mathit{put}\left(r'_2, \left\{ a_1 \right\}, r_1, \mathit{true}, \mathit{A}\right)$} (ESAS);
  \end{tikzpicture}
  \caption{Workflow for wrapping assets with presence conditions, relevant for patterns \emph{P4} and \emph{P5}.}
  \label{fig:patterns:p4}
\end{figure}
These edit operations are for the cases where an existing asset without a
presence condition needs to get a presence condition, and another asset is
created with the negation of that presence condition. As with the previous
system, we can support this workflow by applying \emph{get} with the trivial
expression \(\truefeature\), to then edit that asset and apply the \emph{put}
operator with the ambition set to the desired presence condition of the
edited line. This process can also be seen in Figure~\ref{fig:patterns:p4}.

\subsection*{P6 AddNormalCode}
In this edit operation, no changes are made to variability. In this case, we
should use the \emph{get} operator with as choice the presence condition of the
code we want to edit (this might be the trivial choice.) After editing we
should again use the same ambition as choice to ensure we do not change any
presence conditions.

\subsection*{P7 AddAnnotation}
This edit operation is not relevant to us, it is about repairing broken 
preprocessor annotations. We do not work with \texttt{\#ifdef} statements,
however.

\begin{figure}
  \centering
  \begin{tikzpicture}
    \treetwo{OS}{0}{0}{$r_1$}{$a_1$}{$a_2$}{$A$}{$B$}
    \treetwo{OV}{0}{-2.5}{$r_2$}{$a_1$}{$a_2$}{$A$}{$B$}
    \treeone{EV}{3}{-2.5}{$r'_2$}{$a_1$}{$A$}
    \treeone{ES}{3}{0}{$r_3$}{$a_1$}{$A$}
    \path[] (OVAE) edge[double,-Stealth] node[midway, below] {Manual edit} (EVAW);
    \path[] (OSAS) edge[double,-Stealth] node[left] {$\mathit{get}\left(r_1, \mathit{true}\right)$} (OVAN);
    \path[] (EVAN) edge[double,-Stealth] node[right] {$\mathit{put}\left(r'_2, \left\{ a_2 \right\}, r_1, \mathit{true}, \mathit{true}\right)$} (ESAS);
  \end{tikzpicture}
  \caption{Workflow for removing assets, relevant for patterns \emph{P8} and \emph{P9}.}
  \label{fig:patterns:p8}
\end{figure}

\section{Code-Removing Patterns}
\subsection*{P8 RemNormalCode, P9 RemIfdef}
These patterns cover removing normal code, and code \emph{with} preprocessor
annotations. For us, there is no difference, as the presence conditions are
embedded in the assets. To cover this pattern, we can apply the
\emph{get} operator with any choice that is satisfiable with the presence
condition of the target asset. Then we can delete the asset to finally apply
the \emph{put} operator with any ambition such that the old presence condition
in conjunction with the negation of the ambition is not satisfiable. Using as
ambition \(\truefeature\) always works. A visualisation of this pattern can be
seen in Figure~\ref{fig:patterns:p8}.

\subsection*{P10 RemAnnotation}
Similarly to \emph{P7}, this is not a relevant operation for us. This edit
operation is about removing annotations that were unintentionally left by
removing some other code. These ill-formed annotations cannot occur in our
system as the presence conditions are embedded in the assets.

\section{Other Patterns}
\begin{figure}
  \centering
  \begin{tikzpicture}
    \treetwo{OS}{0}{0}{$r_1$}{$a_1$}{$a_2$}{$A$}{}
    \treetwo{OV}{0}{-2.5}{$r_2$}{$a_1$}{$a_2$}{$A$}{}
    \treeone{EV}{3.5}{-2.5}{$r'_2$}{$a_1$}{$A$}
    \treetwo{ES}{3.5}{0}{$r_3$}{$a_1$}{$a_2$}{$A$}{$A$}
    \path[] (OVAE) edge[double,-Stealth] node[midway, below] {Manual edit} (EVAW);
    \path[] (OSAS) edge[double,-Stealth] node[left] {$\mathit{get}\left(r_1, \mathit{true}\right)$} (OVAN);
    \path[] (EVAN) edge[double,-Stealth] node[right] {$\mathit{put}\left(r'_2, \left\{ a_2 \right\}, r_1, \mathit{true}, \neg\mathit{A}\right)$} (ESAS);
  \end{tikzpicture}
  \caption{Workflow for giving an asset a non-trivial presence condition, relevant for pattern \emph{P11}. In this case, we want to give $a_2$ a presence condition of $A$.}
  \label{fig:patterns:p11}
\end{figure}
\subsection*{P11 WrapCode}
This pattern wraps non-variational code (with a trivial presence condition)
with some presence condition. The way we can do this is by obtaining the asset
with some choice (for example, the trivial choice), removing the asset, to then
\emph{put} the changes back using an ambition equal to the negation of the
desired presence condition. Deleting an asset makes the presence condition of
the original asset a conjunction with the negation of the ambition. This pattern
can be seen in Figure~\ref{fig:patterns:p11}.

\subsection*{P12 UnwrapCode}
In contrast to the previous pattern, this pattern removes the presence condition
from an asset. That is, we want to change the presence condition such that it
becomes the trivial one. We can do this by trying to add a new asset with the
negation of the asset its presence condition. But the better way to support
this is to remove the presence condition from the asset manually, since
adding an asset with the negation of the presence condition requires copying
and pasting the same code twice.

\subsection*{P13 ChangePC}
Changing the presence condition is an operation that is limited in our system.
It can be like \emph{P11}, where we ``added'' onto the old presence condition,
then we can delete the asset and use the negation of what we want to add to
the presence condition as the ambition. But we cannot remove any parts of a
presence condition by applying the \emph{put} operator.

\subsection*{P14 MoveElse}
Moving the \texttt{else} part of a preprocessor annotation is not a sensible action
to us. Since we are not working with literal \texttt{\#ifdef} annotations. For
us, moving the \texttt{else} part is nothing more than changing multiple presence
conditions. Which we have covered in the previous edit operations.

\section{Comparison}
We will now compare the applicabilities of the edit operations above to the previous
method. The previous method uses the exact same choices and ambitions as we did to
``implement'' these operations. This means that even with the restriction of the
\emph{put} operator, we can still support all of the edit patterns. The first
conclusion we can thus draw is that our method has at least the level of applicability
of the previous method. The fact that we can apply the same choices and ambitions to
reach the same goals should not be surprising. As we noted on before, we do not
principally create a new method. The current work and the work that this is based on
have operators with equal goals and (nearly) equal parameters. We \emph{want} them to
do the same, but we aimed to loosen restrictions of the previous work. 

A specific goal of this work is to loosen the restriction of the \emph{ambition} expression,
in comparison to the previous method. To do this, we needed the \emph{put} operator
to not be reliant on the \emph{choice} expression. We know that this is not possible,
hence the need for the restriction on the \emph{put} operator. But we still see an
improvement in terms of applicability, which was already visible in Figure~\ref{fig:example:put}.
The idea is that our definition of \emph{put} only uses the \emph{choice} expression
in internal checks. When applying changes, it only uses the \emph{ambition} expression.
This is in contrast to the previous system, where new \texttt{\#ifdef} statements were
created with both the \emph{choice} and the \emph{ambition}. In particular, this means
that we can create a new feature while hiding certain other features from the view,
as seen in the figure. If we were to apply the same strategy in the previous method, the
new code would be added with as presence condition \(\mathit{GUI} \land \neg\mathit{CLI}\).
This means that the new feature can only be enabled given that the \emph{CLI} feature is not.
This same idea can be used in other editing patterns, which leads to the fact that the 
choice used to obtain the views in the patterns \emph{P1}-\emph{P5}, \emph{P8}, \emph{P9}
and \emph{P11}-\emph{P13} are completely free to the user, as long as the needed assets
are obtained by them (their presence condition is satisfiable with the choice).
Of course, the edit operations still have to admit to the restriction in the 
\emph{put} operator. We will discuss the effects of this limitation in greater detail in 
the next subsection. 

The biggest limitations we came across are most noticeable in patterns
\emph{P12} through \emph{P14}, which were equally limited in the previous system.
We should note that when working with the Virtual Platform, a separate operator 
to change the presence condition would be able to tackle these patterns. A final
important difference with the previous system is that the new system is not (always)
fully automatic: the user sometimes needs to choose the location of a new asset 
when an Alignment Conflict has occurred. This can happen when certain assets are 
not present in the view.

\section{Put Limitation}
The limitation we introduced for the \emph{put} operator in
Section~\ref{sec:configlenses:vp} ensures that the \putget~law is satisfied.
In all operations we do with the \emph{put} operator, we have to make sure that
this limitation is met. This of course limits the operations we can do. We now
discuss the severity of this limitation.

Again, the limitation is as follows, for every asset $a$ in the view, the
following must hold:
\[
  \mathit{choice} \land \mathit{ambition} \land \presenceconditionfunc(a) \in \sat
\]
Some parts of this limitation already logically follow from the \emph{get}
operator. We already know that for every asset $a$ in the view, \(\mathit{choice} \land \presenceconditionfunc\left(a\right) \in \sat\). 
This is by definition of the \emph{get} operator -- if this would not hold, the
asset would not be in the view. At this point we can split the limitation up
into two parts. Firstly, we must make sure that \(\mathit{choice} \land \mathit{ambition} \in \sat\),
secondly, we must make sure that for every asset $a$ in the view, \(\mathit{ambition} \land \presenceconditionfunc\left(a\right) \in \sat\).
The first part is a single expression that we must satisfy. It limits us in
such a way that we cannot have an ambition that is the negation of the choice.
We can still solve those situations, however. By deleting or editing
existing assets, the existing assets get the negation of the ambition added to
their presence conditions. So if we were to choose the ambition equal to the
choice, then the negation of the choice is -- in a way -- used as the ambition.
The second limitation then is that every asset has to be satisfiable with the
chosen ambition. This can lead to some difficulties, we might get the view
with the trivial choice (\(\truefeature\)) and edit some assets. If we want
to \emph{put} these changes back with some ambition, all assets, even those not
edited, must satisfy that their presence condition is satisfiable with this
ambition. In other words: the presence conditions of the assets in the view
must \emph{all} be satisfiable with the ambition expression. There are two ways
to avoid this: we can either apply the \emph{get} operator to an asset such that
the assets breaking this limitation are out of the view, or we can choose a
\emph{choice} in the \emph{get} operator such that it already includes the ambition.

Since this last restriction seems rather strict, we wonder how the previous
method in~\cite{stuanciulescu2016} managed to overcome this issue. In this
process, we figured that their method does not satisfy the \putget~law.
With our ambition restriction applied to their method, this problem is solved. An
example showing how the previous method does not adhere to the \putget~law can
be found in Figure~\ref{fig:prevsystem:counterexample}. Here, we have two
preprocessor annotations, which we check out using the trivial choice
\emph{true}. We then edit the line covered by the first annotation to finally
recombine the changes with as ambition \emph{A}. After minimisation, we get to
the code shown in the top-right. Now the \putget~law says that we should be
able to go back to this view using the \emph{get} operator with the initial
expression and the ambition as the choice expression. In this case, this is just
\emph{A}. But then we get to the problem. When we apply the \emph{get} operator
with as choice \emph{A}, we do not get this view, but rather just the line that
we have edited.

\begin{figure}
    \centering
    \begin{tikzpicture}
        \node[draw,align=left] (OS) at (0,0) {\texttt{\#ifdef A} \\ \texttt{  do\_something()} \\ \texttt{\#else} \\ \texttt{  do\_something\_else()} \\ \texttt{\#endif}};
        \node[draw,align=left] (ES) at (7,0) {\texttt{\#ifdef A} \\ \texttt{  do\_something\_different()} \\ \texttt{\#else} \\ \texttt{  do\_something\_else()} \\ \texttt{\#endif}};
        \node[draw,align=left] (OV) at (0,-4) {\texttt{\#ifdef A} \\ \texttt{  do\_something()} \\ \texttt{\#else} \\ \texttt{  do\_something\_else()} \\ \texttt{\#endif}};
        \node[draw,align=left] (EV) at (7,-4) {\texttt{\#ifdef A} \\ \texttt{  do\_something\_different()} \\ \texttt{\#else} \\ \texttt{  do\_something\_else()} \\ \texttt{\#endif}};            

        \path[] (OV) edge[double,-Stealth] node[midway, below] {Manual edit} (EV);
        \path[] (OS) edge[bend left=-30,double,-Stealth] node[left] {$\mathit{get}\left(\cdot, \mathit{true}\right)$} (OV);
        \path[] (ES) edge[bend left=-30,double,-Stealth,dashed,red] node[left, black] {$\mathit{get}\left(\cdot, \mathit{A}\right)$} (EV);
        \path[] (EV) edge[bend right=30,double,-Stealth] node[right] {$\mathit{put}\left(\cdot, \cdot, \mathit{true}, \mathit{A}\right)$} (ES);
    \end{tikzpicture}
    \caption{Example showing how the system in~\cite{stuanciulescu2016} does not adhere to the \putget~law. The edited view (bottom-right) does not follow from the red edge
             coming from the edited source (top-right). Note that this same limitation applies to our system, hence the restriction we applied on the \emph{put} operator.}
    \label{fig:prevsystem:counterexample}
\end{figure}

A final note on the limitation in terms of the implementation in the Virtual
Platform is that it is not a necessary limitation in terms of usability. The
restriction is purely there for the \putget~law. If the user wants to ignore
this law, they could still apply the \emph{put} operator without satisfying the
law. It just might not be possible to extract that exact view from the source
again. In the implementation in the Virtual Platform, this is done using an
error that can be configured to be a warning instead.
