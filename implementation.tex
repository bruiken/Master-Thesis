\chapter{Implementation}\label{ch:implementation}
In this chapter, we will present the implementation of the operators in the
Virtual Platform. The implementation of the Virtual Platform is not a one-to-one match
of the formalisation created in this work, our formalisation abstracts away some details
that are present in the implementation. We aim to give an overview of our implementation.
After the overview, we go into more detail of selected, particularly interesting parts of the implementation.

The implementation itself is written in Scala\footnote{\url{https://www.scala-lang.org/}} and
can be found in the source code of the Virtual Platform\footnote{\url{https://bitbucket.org/easelab/vp/src/master/}},
which is open source. The Virtual Platform contains many different operators, the \emph{get}
and \emph{put} functions are thus added as two new operators. 

\section{\emph{Get} Operator}
\begin{lstfloat}
\begin{lstlisting}[language=Scala,numbers=left,showstringspaces=false]
override def perform(): Boolean = {
  if (getImmediateAncestorFeatureModel(Some(asset)).isEmpty) {
    log.error("Cannot apply operator on an asset without a feature model")
    return false
  }
  
  val assetClone = asset.doClone()
  
  var parentFM: Option[FeatureModel] = None
  if (assetClone.parent.isDefined) {
    parentFM = assetClone.parent.get.foldFeatureModels()
  }
  if (assetClone.featureModel.isDefined && parentFM.isDefined) {
    assetClone.featureModel.get.crossTreeConstraints :+= parentFM.get.asExpression
  } else if (assetClone.featureModel.isEmpty) {
    assetClone.featureModel = parentFM
  }
  
  if (assetClone.parent.isDefined) {
    assetClone.featureModel.get.crossTreeConstraints :+= assetClone.parent.get.foldPresenceConditions()
  }
  
  assetClone.featureModel.get.crossTreeConstraints :+= choice
  
  val result = applyChoice(assetClone)
  
  if (result.isEmpty) {
    log.warn("The given choice resulted in an empty asset tree")
    false
  } else {
    val root = asset.getRoot()
    if (root.isDefined) {
      root.get.unassignedAssets :+= result.get
    }
    true
  }
}
\end{lstlisting}
\caption{Implementation of the \emph{get} operator}
\label{lst:getoperator}
\end{lstfloat}

In Listing~\ref{lst:getoperator}, we can see the source code for the \emph{get} operator.
The first check in the operator is the availability of a feature model somewhere in the
ancestors of the asset that the operator is applied on. We need this to save the folded
feature models and presence conditions. This is also a difference between the implementation
and the formalisation: the formalisation sees the feature model as an expression, while the
implementation sees the feature model as a tree of features, together with a list of
expressions that can be used as cross tree constraints. We want to access the cross-tree
constraints and thus need some feature model to save them into. This check can be found in
lines 2-5. 

To continue, we first clone the target asset, such that we can use it to change it without
changing the original asset. This cloning is done using the \texttt{.doClone()} method in
line 7.

Next, we fold the parent feature models using the \texttt{.foldFeatureModels()} method, this
method results in an optional new feature model. We want to save this folded feature model
in either the cross tree constraints of the feature model of the current asset, or as the
feature model of the current asset in general. This is depending on if the current asset
already has a feature model. This is done in lines 9-17. The same is then done for the 
presence conditions in the ancestors of the current asset. This time we do not have to
check if the current asset has a feature model, as we already know this from the previous
step. The folded presence conditions are saved in lines 19-23.

Lastly, the actual pruning of the assets is done using the \texttt{applyChoice()} method.
We will cover this in more detail next, assume for now that this method results in an 
optional asset (line 25). Then in lines 27-36 the result is parsed. If there is no result
(the target asset was not included in the result), the user is warned about this. Otherwise,
we save the new asset in the \emph{unassignedAssets} of the root asset. This is another
difference between the implementation and the formalisation, we want to keep the full
tree connected in the implementation. To do this, we have to include the cloned asset in
the original asset tree. This is done by attaching the cloned assets to the \emph{unassignedAssets}.

\begin{lstfloat}
\begin{lstlisting}[language=Scala,numbers=left,showstringspaces=false]
private def applyChoice(localAsset: Asset): Option[Asset] = {
  var expression = localAsset.presenceCondition & choice
  if (localAsset.featureModel.isDefined) {
    expression &= localAsset.featureModel.get.asExpression
  }
  if (SATSolver.solve(expression) == UNSAT()) {
    None
  } else {
    val cloneAsset = localAsset.doClone()
    cloneAsset.children = localAsset.children.flatMap(applyChoice)
    Some(cloneAsset)
  }
}
\end{lstlisting}
\caption{Implementation of the \emph{applyChoice} method}
\label{lst:getoperator:applychoice}
\end{lstfloat}

\subsection{The \emph{applyChoice} Method}
The workings of the \emph{get} operator have been explained in the previous section,
but this explanation skips over the most important function of the operator. This
function actually does the filtering of the assets to decide which ones should end up
in the result. It is called \emph{applyChoice} and its implementation can be found in
Listing~\ref{lst:getoperator:applychoice}.

The function takes one argument, which is the asset that we are currently looking at
(\emph{localAsset}). We first want to create the expression which we want to check for
satisfiability, we do this by creating a conjunction with the \emph{choice}, the local
presence condition, and the local feature model. The feature model is optional, so we
have to do an extra check before using it. This expression is built in lines 2-5.

Next, we actually apply a SAT solver to check the expression for satisfiability, for
this, we use the SAT solver available in the Virtual Platform. If this is not satisfiable
(the result is \emph{UNSAT}), we return \emph{None} to indicate that the asset should
not appear in the result asset set. If the expression was satisfiable, we want to return
the asset. Before returning it however, we want to apply the filtering function on the children
of the asset. 

\section{\emph{Put} Operator}
\begin{lstfloat}
\begin{lstlisting}[language=Scala,numbers=left,showstringspaces=false]
override def perform(): Boolean = {
  if (!doSanityChecks()) return false
  filterChangedAssets()

  for (changedAsset <- changedAssets) {
    if (targetAssetMap.contains(changedAsset)) {
      targetAssetMap(changedAsset).presenceCondition &= !ambition
    }
    if (sourceAssetMap.contains(changedAsset)) {
      sourceAssetMap(changedAsset).presenceCondition &= ambition
      insertNewAsset(sourceAssetMap(changedAsset))
    }
  }

  val targetRoot = targetAsset.getRoot().get
  targetRoot.unassignedAssets = targetRoot.unassignedAssets filterNot {_.name == asset.name}

  true
}
\end{lstlisting}
\caption{Implementation of the \emph{put} operator}
\label{lst:putoperator}
\end{lstfloat}
The main function of the implementation of the \emph{Put} operator closely
matches the definition we gave in pseudocode in Algorithm~\ref{alg:vp:put}.
The operator starts by doing a number of sanity checks on the assets on which
the operator was applied. These checks make sure that the changed assets are
valid, the target asset is actually the source of the new asset, if the
new asset is correctly created using the \emph{get} operator, and finally if
the new asset tree satisfies the limitation we created for the \emph{put}
operator. This is done in line 2.

In the next line, we filter the changed assets, such that changed assets
that are children of another changed asset are removed from the list (line 3).
The following for loop enumerates the changed assets and applies the necessary
changes to the original asset tree. At this point it should be noted that even
before calling the \emph{put} operator, two maps are created to easily access the
old and new assets from their respective asset trees. These are called the 
\emph{targetAssetMap} for the original asset tree and \emph{sourceAssetMap} for
the new asset tree. These maps can be used to check if the changed assets are
edited, created or deleted. The necessary changed are made to the presence conditions
(adding the ambition or the negation of the ambition) and the new asset is inserted
in the asset tree using the \texttt{insertNewAsset} method in case this is needed.
We will go into more detail of this function later. This process can be found in
lines 5-13. Lastly, we remove the cloned asset tree from the \emph{unassignedAssets}
field of the target asset tree, as the \emph{get}-\emph{put} cycle is now complete
(lines 15-16).

\begin{lstfloat}
\begin{lstlisting}[language=Scala,numbers=left,showstringspaces=false]
private def insertNewAsset(newAsset: Asset): Unit = {
  if (targetAssetMap.contains(newAsset.name)) {
    val oldAsset = targetAssetMap(newAsset.name)
    val oldAssetIndex = oldAsset.parent.get.children.indexOf(oldAsset)
    oldAsset.parent.get.children = Put.insertInList(oldAsset.parent.get.children, newAsset, oldAssetIndex + 1)
  } else if (Put.childrenAreOrdered(newAsset.parent.get.assetType)) {
    val targetParent = targetAssetMap(newAsset.parent.get.name)

    val lowerBound = ...
    val upperBound = ...
    var newIdx = -1
    if (lowerBound == upperBound) {
      newIdx = lowerBound
    } else {
      do {
        print(f"Please select an index for asset ${newAsset.name}")
        newIdx = scala.io.StdIn.readInt()
      } while (newIdx > upperBound || newIdx < lowerBound)
    }
    targetParent.children = Put.insertInList(targetParent.children, newAsset, newIdx)
  } else {
    val targetParent = targetAssetMap(newAsset.parent.get.name)
    targetParent.children = Put.insertInList(targetParent.children, newAsset, idx = 0)
  }
}
\end{lstlisting}
\caption{Implementation of the \emph{insertNewAsset} method}
\label{lst:putoperator:insertnewasset}
\end{lstfloat}

\subsection{The \emph{insertNewAsset} Method}
Before, we have seen the general structure of how the \emph{put} operator
is implemented. Let us now look at the most important function of the implementation,
namely the one that actually inserts the new assets into the source asset tree.
This function is called \emph{insertNewAsset}, it takes as an argument the asset to add
(called \emph{newAsset}). Its implementation can be found in Listing~\ref{lst:putoperator:insertnewasset}.

The method is mainly split up into three cases: the first case is an edit operation, in
which we already know where to place the new asset (lines 2-5). The second case is the case where
it is an addition of a new asset, where the ordering is important (lines 6-20). The last case is a
new asset where the ordering does not matter (lines 21-24).

\subsubsection*{Edit Operation}
The first case is an edit operation of an asset. We have a new asset to add, and we want to
place it right beside the asset that was edited. An asset can be added using the
\texttt{insertInList} method, for which we only need the new index of the asset.
To get the index, we first get the original of the edited asset in line 3. With this asset,
we can decide the index using the \texttt{indexOf} method in line 4. The insertion in the list
of children is then done in line 5.

\subsubsection*{Ordered Addition Operation}
As mentioned in the formalisation, when an asset is added and we have multiple choices
for the new location, we have to ask the programmer where they want to place the new
asset. We first decide the lower- and upper bounds for the location in lines 7-11. If 
these bounds are equal, we have only one possible location and do not have to ask the
programmer for help (lines 12-13). Otherwise, we ask the programmer for help (lines 14-19).
Note that the message the programmer gets is simplified in the listing shown here. Lastly,
we insert the new asset in the chosen location in line 20.

\subsubsection*{Unordered Addition Operation}
Lastly, we have the unordered addition operation. This is used for assets that do not have
to be ordered (such as files). Here, we simply add the new asset to the first position of
the list of children. This is the simplest variant as no index calculation has to be done.
We can simply get the target parent and add the new child to it (lines 21-24).

\section{Implementation Remarks}
In this chapter, we have seen the major parts of the implementation of the operators.
One last remark that should be made is the shortcoming of the current implementation.
The Virtual Platform of course does not work without serialisation and deserialisation
of the code that is represented using the asset trees. Without these steps, editing the
code is near impossible. We have not implemented the (de)serialisation of the cloned
asset trees, because the Virtual Platform currently cannot (de)serialise presence
conditions that are not disjunctions of features. We need more complex presence conditions
to support our operators. Before support is created for (de)serialisation of presence
conditions in the shape of any expressions, our created operators are not fully finished.
The core logic is still implemented and it can still be used as a proof of concept, simply
without actually having the source files generated.

\begin{figure}
    \centering
    \begin{tikzpicture}
        \node[draw,align=left] (ES) at (7,0) {\texttt{Root()} \\ \mbox{\texttt{~Repo()}} \\ \mbox{\texttt{~~File()}} \\ \mbox{\texttt{~~~c1 [!GUI]}} \\ \mbox{\texttt{~~~c1 [GUI]}} \\ \mbox{\texttt{~~~c4 [GUI]}} \\ \mbox{\texttt{~~~~m1()}} \\ \mbox{\texttt{~~~c2 [CLI]}} \\ \mbox{\texttt{~~~c3 [logging]}}};
        \node[draw,align=left] (OS) at (0,0) {\texttt{Root()} \\ \mbox{\texttt{~Repo()}} \\ \mbox{\texttt{~~File()}} \\ \mbox{\texttt{~~~c1()}} \\ \mbox{\texttt{~~~c2 [CLI]}} \\ \mbox{\texttt{~~~c3 [logging]}}};
        \node[draw,align=left] (OV) at (0,-4.5) {\texttt{Repo()} \\ \mbox{\texttt{~File()}} \\ \mbox{\texttt{~~c1()}}};
        \node[draw,align=left] (EV) at (7,-4.5) {\texttt{Repo()} \\ \mbox{\texttt{~File()}} \\ \mbox{\texttt{~~c1()}} \\ \mbox{\texttt{~~c4()}} \\ \mbox{\texttt{~~~m1()}}};

        \path[] (OV) edge[double,-Stealth] node[midway, below] {Manual edit} (EV);
        \path[] (OS) edge[bend left=-30,double,-Stealth] node[left] {$\mathit{get}\left(\cdot, \neg\mathit{CLI}\right)$} (OV);
        \path[] (EV) edge[bend right=30,double,-Stealth] node[left] {$\mathit{put}\left(\cdot, \left\{ c_1, c_4, m_1 \right\}, \cdot, \neg\mathit{CLI}, \mathit{GUI}\right)$} (ES);
    \end{tikzpicture}
    \caption{Example~\ref{example:putoperator} recreated using the Virtual Platform.}
    \label{fig:implementation:example}
\end{figure}

\begin{example}[Continuation of Example~\ref{example:putoperator}]
  In this example, we will recreate the example of the \emph{put} operator in
  the implementation in the Virtual Platform. The result of the previous
  example of the \emph{put} operator can be found in Figure~\ref{fig:example:put}.
  The visualisation that the Virtual Platform gives is seen in Figure~\ref{fig:implementation:example}.
  Here, a tree structure is shown of how the assets relate to each other. An identation
  means that the asset is a child asset of the asset above. The name of the asset is shown
  together with an optional presence condition between square braces. 
  It can be seen that this represents the same structure as seen in the previous example
  using the \emph{put} operator from the formalisation.
  The only thing that the Virtual Platform asked of us is that we have to give a 
  location for $c_4$, as it can be put it multiple places: this is the alignment conflict.
  The way the Virtual Platform asks this is as follows:
  \begin{verbatim}
An alignment conflict has occurred!
1-> c1 <-2-> c2 <-3
Please select an index for asset c4 (within the range [1, 3]):
  \end{verbatim}
  This represents the children of the \emph{File} asset, we see $c_1$ and $c_2$ together
  with numbers pointing to them. We can now choose either $1$, $2$ or $3$ to select a
  location: $1$ to place the new asset \emph{before} $c_1$, $2$ to place it \emph{after}
  $c_1$, and $3$ to place it \emph{after} $c_2$. We choose $2$ here, to adhere to the example.
\end{example}
