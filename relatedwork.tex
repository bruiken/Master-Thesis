\chapter{Related Work}\label{sec:relatedwork}
\section{View-Based Editing}
As noted above, the most closely related work is that of St{\u{a}}nciulescu
et al., where a projection-based variation control system is designed~\cite{stuanciulescu2016}.
For this, they use so-called \emph{Choice Calculus}~\cite{walkingshaw2012} to
define the system, a formal language to model variation in software.
This work relieved the restrictions on the \emph{put} operator from previous
work of Walkingshaw and Ostermann~\cite{walkingshaw2014} by using a looser
definition of the \emph{Edit-Isolation Principle}. The principle says that 
the \emph{put} operator cannot affect code outside of the view given by the
\emph{get} operator. The work of Walkingshaw and Ostermann did not make use
of an ambition operator yet, which avoids the complications that we deal with 
in our work. For this setup, they proved the lense laws, as the present work 
does for the broader scope of view-based editing with ambitions.

There is a lot of research into easing programming in variant-rich software.
In the work of K{\"a}stner et al., an IDE is created to visualise software
variabilities in the source code. To this end, they use a tool called
\emph{CIDE} which was created in earlier work ~\cite{kastner2008, kastner2008granularity}.
They aim to help developers to understand and explore individual features.
Another visualisation tool similar to \emph{CIDE} is created by Heidenreich
et al.. MappingViews is a tool that visualises different views of variational
software to the programmer~\cite{heidenreich2008}. Nestor et al. present more
work on visualisation which scales better to medium and large software product
lines~\cite{nestor2007visualisation}. These tools are limited to visualisation,
they do not deal with the view-update problem.

Kersten and Murphy created an Eclipse plugin called Mylar that uses task
contexts to improve programmer productivity~\cite{kersten2006using}. Their
plugin stores structural relationships of programs in so-called task contexts.
Programmers can then use these contexts to quickly swap between contexts. This
tool is more of a tracing tool and does not work on the view-update problem.

This research direction has roots in earlier research on software maintainability.
Weiser has shown in 1984 how \emph{Slicing} can help understand programs~\cite{weiser1984program}. With this
approach, it is possible to select several variables and see the flow of
the data in this program. In the same year, Linton created a way to store
programs in a database such that different views, cross-sections or slices
could easily be extracted or updated~\cite{linton1984}. These tools are also
for visualisations.

Chu-Caroll et al. created a system called VSC, which stands for visual
separation of concerns, this system can provide certain views of the source
code in terms of features~\cite{chu2003visual}. VSC, like the Virtual Platform,
stores programs using finer-grained artefacts. They do this on a storage-based
level, such that it becomes language independent. Hofer et al. try to tackle
the problem of multiple views at the filesystem level using the Leviathan
filesystem~\cite{hofer2010toolchain}. The idea is that other tools all need
special development environments, by having variant views at the filesystem
level, other tools can be used without issues. While VSC does not work with
write-back (the view-update problem), the Leviathan filesystem can do this.
It is however limited to changing only non-variational code. Aside from that, it sometimes
uses heuristics to decide where changed code should go in the source.

More in terms of variability management, we of course have the Virtual
Platform, which aims to bridge the gap between \emph{Clone \& Own} and a full
\emph{Integrated Platform}~\cite{mahmood2021}. The Virtual Platform is relevant
to us, as we created our implementation onto it. Similar to the Virtual Platform,
research has also been done on tools to transition from \emph{Clone \& Own} to an
integrated platform, for example by Schw{\"a}gerl et al. in SuperMod~\cite{schwagerl2015}.
ECCO is a framework and tool for easing the work on variational software created
by Fischer et al.. It can find common parts and create new software variants
using those parts~\cite{fischer2014enhancing}. 

\section{Lenses}
Lenses were introduced by Foster et al. as a solution to the view-update
problem that originates from databases~\cite{foster2007}. Here, lenses were
formalised and lens laws were created.

Many different types and classes of lenses have been created in the meanwhile.
All of the lenses noted here differ from the first definition of lenses but
are not able to carry configuration data. Hence the need for us to create \emph{Variational Lenses}.
There is a distinction between Symmetric and Asymmetric lenses~\cite{hofmann2011symmetric}.
Asymmetric lenses usually have a source and a view, where the view contains a
subset of the information of the source. Symmetric lenses, on the other hand,
have two sets that both contain some information that the other set does not
have. 

Boomerang, a language created by Bohannon et al., can create lenses that work
between string data~\cite{bohannon2008boomerang}. The user creates a lens by
writing it in the specification of Boomerang and can then use it to transform
strings. The language is based on regular expressions. Boomerang also contains
logic to deal with alignment problems within the data. To this end, Matching
Lenses were created~\cite{barbosa2010matching} by Barbosa et al.. Matching
lenses expand on the earlier definition of Dictionary Lenses which were also
included in Boomerang in~\cite{bohannon2008boomerang}. These lenses make use of
extra data structures to save the ordering of the data, this data structure can
then be used in the definition of the \emph{put} operator. Miltner et al. have
created a way to synthesise Bijective Lenses from examples of in- and output.
The resulting lens is defined in the Boomerang language~\cite{miltner2017synthesizing}.
Bijective lenses differ from the standard lenses in that they do not hide data
going from the source to the view, rather they only change the structure of it.
Another extension to Boomerang is Quotient Lenses, created by Foster et al..
Quotient lenses can contain extra functions to normalise input or output data~\cite{foster2008quotient}.
This way, spacing or newlines can be normalised before going through the
\emph{get} and \emph{put} functions.

Monadic Lenses deal with the fact that most lenses are pure, while most
programming languages have side effects. In \cite{abou2016reflections},
different monadic lenses are discussed.

A more applied version of a lens is for example Lenses for Web Data~\cite{rajkumar2014lenses}.
Here, lenses are implemented to create forms for web pages and to obtain the
data from them. This is implemented in Haskell.

An example of symmetric lenses are Edit Lenses, created by Hofmann et al.. Edit
lenses work by transforming between representations using descriptions of
changes instead of the entire views~\cite{hofmann2012edit}. 

Our new \emph{Variational Lenses} can be seen as asymmetric lenses, as we only
hide data in one direction (from the source to the view). We also take some 
inspiration from edit lenses, however, as we use a ``diff'' structure in the \emph{put}
operator. This structure can be seen as the descriptions of changes made, but
in contrast to edit lenses, these structures by themselves are not sufficient
to make the operator work.
