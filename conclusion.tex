\chapter{Conclusion}\label{sec:conclusion}
In this work, we created and formalised a lens for view-based editing. From 
this formalisation
specifically for the Virtual Platform, we derived a more general type of lens,
which we call \emph{Variational Lenses}. This lens differs from other lenses
since it can carry configuration with it and enables it to be a framework for
view-based editing. In the evaluation, we have shown that the system
can comply with all common edit operations that were previously established. 
We have also shown that this new definition allows for more general use than 
the most closely related previous work.
In particular, we can create and edit
independent features in a limited view, which was not possible in the previous work. 
The downside of the system is that we had to include a limitation to the 
\emph{put} operator of the lens, since without it, one of the lens laws would
not hold. We also established that this lens law does not hold in the previous
system without this new restriction. Finally, we implemented the lens in the
Virtual Platform as two new operators.

The definition of the Variational Lenses was extracted from the formalisation
of the \emph{get} and \emph{put} operators. With this new formalisation, we had
to adapt the existing lens laws for our new lens and we have proven that our
implementation of the lens adheres to these adapted lens laws.

There are several possible directions for future work. First, it would be worthwhile
to study the usability of our proposed operations using a user study, 
in which developers could solve some program evolution tasks in a certain view and use 
our framework to keep the view synchronized with the overall platform. Second, while 
we achieved our results in the context of a specific variability management system 
(Virtual Platform), it would be interesting to further increase the scope of our 
contribution by investigating alternative program representations.
